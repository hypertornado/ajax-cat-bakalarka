
\chapter{Překlad a strojový překlad}

\section{Překladové problémy}

Překlad je proces přenesení významu z textu ve zdrojovém jazyce do jazyka cílového. Úloha překladu je složitá i tím, že žádný výsledek nejde označit za nejlepší. Že neexistuje dokonalý překlad lze ilustrovat na překladech knih, nebo divadelních her. Hry Williama Shakespearea byly z angličtiny do češtiny přeloženy mnohokrát, přesto jsou stále inscenovány hry s různými překlady.

Bez hlubších jazykových znalostí se může jevit úloha překladu snadná, mezi většinou jazyků máme přeci slovník. Ale pokud chceme přeložit anglické slovo "house" do češtiny, můžeme mít problém. Ve většině případů lze toto slovo přeložit jako "dům". Pokud ale překládáme text o anglické královně, kde se objeví sousloví "House of Windsor", zřejmě není řeč o domu, kde bydlí Windsorové, ale o "rodu Windsorů". Překladatel tedy při textu potřebuje znát kontext ve kterém je slovo použito a často také potřebuje mít odborné znalosti z oboru překládaného textu.

Jazyk není neměnný a v průběhu času se vyvíjí. Můžeme to vidět například na Bibli. Její nejznámější překlad, Bible Kralická je přes 300 (?) let starý. Vznikají proto nové překlady, které jsou dnešním čtenářům přístupnější. Žádný překlad tedy nelze označit za dokonalý a navždy správný.

\section{Historie strojového překladu}

Na počátku dějin strojového překladu stála, podobně jako v mnoha jiných oborech, armáda. Spojené státy Americké byli v padesátých letech ve Studené válce se Sovětským svazem. V této válce beze zbraní sehráli velkou úlohu i výzvědné služby, které zachytávali velké množství nepřátelských zpráv. Tyto zprávy bylo nutné co nejrychleji přeložit. A právě v této době se zrodila myšlenka použít k tomuto účelu počítače, které byli produktem předchozího válečného konfliktu, 2. světové války. ( http://www.hutchinsweb.me.uk/GU-IBM-2005.pdf ) Významnou demonstrací použití strojového překladu se v roce 1954 stal takzvaný Georgetownský experiment. Pro tento experiment vyvinula Georgetownská univerzita spolu s firmou IBM překladový systém pro překlad z ruštiny do angličtiny. Tento systém používal slovník 250 slov a 6 gramatických pravidel. Jeho doménou byly zejména překlady v oblasti chemie. Během experimentu bylo přeloženo více než 60 vět. Experiment byl všeobecně přijat jako úspěch, což donutilo americkou vládu investovat v následujících letech do oblasti strojového překladu.

Následovaly léta práce zejména v SSSR a USA na systémech pro automatické překlady zejména mezi ruštinou a angličtinou. Žádný dobře použitelný systém, který by poskytoval uspokojivé výsledky, však nevzniknul. Pochybnosti ohledně možností strojového překladu vyjádřil na konci padesátých let lingvista Yehoshua Bar-Hillel. Ten argumentoval pomocí anglické věty "The box was in the pen." Překlad této věty by mohl být: "Pero bylo v ohradě." Jelikož anglické slovo "pen" znamená "pero" i "ohrada", musí mít překladový systém, který chce větu přeložit správně, sémantickou informaci, která by mu napověděla, že krabice nemůže být peru, tedy že správným překladem slova "pen" do češtiny je v tomto kontextu slovo "ohrada".

http://www.hutchinsweb.me.uk/ALPAC-1996.pdf
I z tohoto důvodu bylo vytvoření komplexního překladového systému v té době zřejmě nemožné. Americká vláda však dále pokračuje ve financování výzkumu. V roce 1966 vyšla zpráva skupiny ALPAC (Automatic Language Processing Advisory Committee). Která doporučovala americké vládě další postup při financování překladového výzkumu. Zpráva zmenšovala optimismus, vyvolaný zejména Georgetownským experimentem, že se v dohledné době podaří vytvořit kvalitní systém pro strojový překlad. Výsledkem bylo téměř úplné zastavení financování výzkumu americkou vládou. Výzkum dále pokračoval zejména v Evropě, nebo Kanadě. Právě v kanadském Montrealu vznikl systém METEO. Ten byl v letech 1981 až 2001 používán pro překlad meteorologických zpráv mezi angličtinou a francouzštinou. Právě omezená překladová doména systému umožnila nabízet kvalitní překlady předpovědí počasí.


\section{Součastnost strojového překladu}
V posledních letech spolu s pokračující globalizací světa a stále vyšší penetrací internetového připojení se zvyšuje i poptávka po překladech. Nadnárodní firmy potřebují při svém růstu stále více překladů. Dalším impulzem zvyšujícím popávku po překladech je i rozšiřování Evropské Unie. V součastnosti unie používá 23 oficiálních jazyků ve kterých musí být přístupné všechny důležité úřední dokumenty. Tvorba tolika překladů je velice pracná a nákladná, což vytváří poptávku po zjednodušením procesu překladu.

Výpočetní výkon počítačů stále roste rychlostí Mooreova zákona (ftp://download.intel.com/research/silicon/moorespaper.pdf), což v posledních letech otevřelo možnosti pro použití statistických metod ve strojovém překladu. Toho využívá statistický překladový systém Moses, open-source překladový systém, který používám i ve svém projektu a cílem projektu byla i implementace nových rozšíření. Dalším známým statistickým překladovým systémem je Google Translate.

Kromě systému využívajících statických postupů se stále vyvíjí i pravidlové překladové systémy. Kromě mnoha proprietálních systémů bych zmínil open-source systém Apertium. Stejně jako u dalších podobných systémů se pro každou dvojici překládaných jazyků musí vytvořit slovníky a překladová pravidla. Je velmi náročné a nákladné tato pravidla vytvořit. Navíc jsou tato pravidla často použitelná pouze pro jeden jazykový pár.

\section{Computer-aided translation}
CAT, neboli computer-aided translation, či computer-assisted translation je zkratka označující systémy pro podporu překladu. Tyto systémy poskytují překladateli podporu při překladu. Mohou to být jak desktopové, tak online aplikace a často se liší stylem, jakým překladatele podporují. Jedním z druhů podpory může být nabídka předchozích překladů z paměti. Této paměti se říká překladová paměť, obsahuje přeložené úseky textu a překladatel si tuto paměť buduje buď sám, nebo může využít nějakou z kolektivních databází. Příkladem desktopové aplikace může být například OmegaT. Ta je určena pro použití profesionálními překladateli, kterým nabízí úseky z překladové paměti. Hledaný úsek nemusí odpovídat aktuálně překládanému úseku na 100 procent, OmegaT implementuje algoritmus, který pozná i blízké shody.

Další ukázkou CAT pomůcky je Google Translator Toolkit (Nástroje pro překladatele). Tato internetová aplikace umožňuje překladateli nahrát si svou překladovou paměť, kterou pak může využít při překladu. Dále nástroj nabízí výsledky překladu z Google Translate, který dále může uživatel editovat.

\section{Moses}

