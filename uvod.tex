\chapter*{Úvod}
\addcontentsline{toc}{chapter}{Úvod}

Díky novým statistickým přístupům zažívá obor strojového překladu jazyka v posledních letech opět velký rozvoj. Nové výpočetní i algoritmické možnosti umožňují vytváření stále lepších jazykových překladů. Stále se však nepodařilo vytvořit univerzální překladový systém, který by dokázal nahradit lidské překladatele, ani v jednom běžném jazykovém páru. Je stále otázka zdali se podobný překladový systém v budoucnosti lidstvu podaří sestavit. Již dnes jsou ale překladové systémy na úrovni, která sice nedokáže překladatele nahradit, ale v mnoha odvětvích usnadňuje jejich práci. Překladové systémy již nyní poskytují alespoň nápovědu, jak daný text přeložit. Překladatel však stále musí výstup z takového systému kontrolovat a editovat. Každý z těchto editačních zásahů představuje pro překladatele komplikaci a pokud je množství nutných zásahů nad nějakou hranicí, překladatel raději místo editování výstupu překladového systému vytvoří překlad sám. Pro zjednodušení překladatelské práce je tedy potřeba nejen zlepšovat tyto překladové systémy, ale také software kteří překladatelé pro interakci s překladovým systémem používají. Překladový software, který využívá pro nápovědu překladatelům strojový překlad je speciální případem CAT (computer--aided translation) systému.

Cílem této bakalářské práce je implementace jednoho CAT systému. Pro podporu překladu bude systém využívat překladový systém Moses. Celý projekt je rozdělen do dvou částí --- serverová a klientská část. Implementací serverové části bude vytvořen HTTP server. Tento server bude spouštět Mosese a skrz HTTP požadavky bude poskytovat klientovi odpovědi. Požadavky budou dvojího typu. Klient se může systému zeptat na překlad věty v určité jazyce. Server pak odpoví tabulkou překladových možností. Sloupce této tabulky jsou jednotlivé úseky ve zdrojovém překladovém úseku (typicky slova ve větě). V řádcích tabulky jsou pak přeložené úseky textu v cílovém jazyce. Úseky jsou seřazeny v tabulce tak, že čím výše je daný úsek, tím větší je pravděpodobnost toho, že se jedná o "správný překlad". Taková to tabulka je jedním ze základů implementovaného CAT systému.

Dalším typem klientského požadavku bude jakási lokální nápověda během překladu. Jedná se o podobný druh nápovědy jakou nám poskytují například intenetové vyhledávače. V nich často uživatel nemusí psát celý vyhledávací dotaz a může využí nápovědy, která mu nabízí nejběžnější podobné dotazy. Podobně i implementovaný server bude dávat nápovědu, jak dále pokračovat s překladem. Aby překladový systém mohl tuto nápovědu poskytnout, pořebuje znát tři parametry. Text ve zdrojovém jazyce, vektor určující, které úseky jsou již přeloženy a již přeložená část věty v cílovém jazyce. Cílem této práce je i rozšířit možnosti Mosese tak, aby s těmito třemi parametry dokázal pracovat a vygeneroval nápovědu, jak v překladu pokračovat.

Samotná serverová část tak bude moci fungovat jako komponenta samostatně a poskytovat nápovědu k překladu i jiném CAT systému.

Druhou částí práce je implementace klientské části CAT systému. Tato část bude sloužit k interakci překladatele s překladový systémem. Tato interakce by měla být co nejvíce přátelská k uživateli. Ten typicky zadá zdrojový text pro překlad a zdrojový a cílový jazyk překladu. CAT systém tento zdrojový text rozseká do bloků (typicky vět) a ke každé větě překladateli nabídne nápovědu generovanou v serverové části. Součástí implementace klientské části bude i jednoduchý systém pro zprávu obsahu, aby překladatel mohl pokračovat v překladu i po znovuotevření aplikace. Klientská část bude podobně jako serverová část fungovat sama o sobě. Pokud tedy nebude napojena na server, může pracovat sama o sobě jako systém pro správu překladů.


